\documentclass[10pt]{beamer}

\usetheme[progressbar=frametitle]{metropolis}
\usepackage{appendixnumberbeamer}

\usepackage{siunitx}
\usepackage{booktabs}
\usepackage[scale=2]{ccicons}

\usepackage{hyperref}

\usepackage{pgfplots}
\usepgfplotslibrary{dateplot}

\usepackage{xspace}
\newcommand{\themename}{\textbf{\textsc{metropolis}}\xspace}

\title{Course Outline}
\subtitle{ASTR288P}
\date{2017/09/01}
\author{Sean Griffin}
\institute{UMCP / NASA GSFC}
\titlegraphic{\vfill\includegraphics[height=1cm]{figures/University_of_Maryland_Seal.png}}

\newcommand{\ie}{\emph{i.e.}}
\newcommand{\eg}{e.g.}
\begin{document}

\maketitle

%\begin{frame}{Table of contents}
%  \setbeamertemplate{section in toc}[sections numbered]
%  \tableofcontents[hideallsubsections]
%\end{frame}

\begin{frame}[fragile]{About Me}
	\begin{itemize}
		\item Office: ATL 0251A
		\item Office hours: Friday 1PM-2PM (before class)
                \item Always confirm with be before coming for office hours!
                \item astro email: sgriffin 
		\item I'm actually based at NASA Goddard, so I'm not around on other days.
	\end{itemize}
\end{frame}

\begin{frame}[fragile]{Course format}
	\begin{itemize}
		\item Homework: 
		
		We have them, they will be assigned as the course progresses ("natural stopping points").
			
		Generally due at the start of the following class.

                There will be time to work on them in-class.
                
		\item End-of-term project and presentation in place of a final written exam.
	\end{itemize}
\end{frame}

\begin{frame}[fragile]{Resources}
	Books: Is that like the Internet but made out of a tree? 
	
	Online: 
	\begin{itemize}
		\item Wikipedia
		\item \url{http://www.stackoverflow.com} Any Questions
		\item \url{http://www.codecademy.com} Python, Git, ...
		\item \url{http://tutorialspoint.com/cprogramming} C language
		\item \url{http://projecteuler.net} Challenging Problems
   		\item \url{http://rosettacode.org/wiki/Averages/Arithmetic_mean}
	\end{itemize}

\end{frame}

\begin{frame}[fragile]{What we'll cover (more advanced topics)}

    \begin{itemize}
    	\item UNIX:
    	\begin{itemize}
    		\item Shell (we will use \textbf{bash} but others exist), shell scripting
    		\item File system (/, /usr/bin, \$HOME, etc.)
    		\item Window managers, desktop environments
    		\item Editors (\textbf{emacs}, vi, gedit, pico, \textbf{sublime}, many others) -- people have strong opinions but I do not!
    		\item Base commands (\textbf{cd, mkdir, ls, ssh}...)
		\item Tools (\textbf{git}, gcc, )
    	\end{itemize}
    	\item Scripting
    	\begin{itemize}
    		\item Python
    		\item ipython
    		\item Jupyter
    	\end{itemize}
	\item Word processing with \LaTeX:
	\begin{itemize}
		\item Not included last time this course was taught but I think it's important to learn early on.
		\item This document produced using \LaTeX.
	\end{itemize}
	
	 \end{itemize}
\end{frame}	

\begin{frame}[fragile]{What we'll cover}
    \begin{itemize}	
    	\item Some Object Oriented Programming (OOP)
    	\item Compiled code
    	\begin{itemize}
    		\item C/C++, Makefiles
    		\item FORTRAN is a thing that exists but we won't use it.
    	\end{itemize}		
    	
    	\item Data analysis
	\begin{itemize}
		\item Compiling and running analysis code
    		\item Analysis and plotting
    	\end{itemize}		
    
    \end{itemize}

\end{frame}

\begin{frame}[fragile]{What we probably wont cover}

	\begin{itemize}
		\item Machine learning
		\item Multi-threaded / parallel programming (OpenMP, MPI, CUDA)
		\item ...
	\end{itemize}

\end{frame}

\begin{frame}[fragile]{Hardware}
	Lab machines:
	\begin{itemize}
		\item Master: ursa.astro.umd.edu
		\item Nodes: lab001, lab002, ... lab013
		\item Printer: labs.astro.umd.edu
	\end{itemize}


	Virtual machines:
	\begin{itemize}
		\item \textbf{virtualbox}
		\item vmware
	\end{itemize}

	Your own machine:
	\begin{itemize}
		\item Linux (Ubuntu, Redhat, Fedora, Mint, debian, gentoo...) -- 	You can dual-boot if you want. 
		\item Mac OS X
		\begin{itemize}
			\item You need Xquartz installed so certain features will work. 
		\end{itemize}
		\item Windows -- Not recommended for this class
		\begin{itemize}
			\item putty, Windows10 bash, VNC viewer
			\item Probably easier just to run a virtual machine (this is what I do!). 
		\end{itemize}
	\end{itemize}

\end{frame}

\begin{frame}[fragile]{Virtualbox}

  Virtualbox is a program that allows you to create and run virtual computeres within your host computer.

  This is a handy alternative to dual-booting if you want to use both Linux and Windows (e.g. if you're a hardcore gamer but need *nix to do analysis for work).

  You are effectively running an entire computer inside your OS, so performance can be an issue (particularly graphical stuff, although things are getting better).

  We'll go through the process of creating a virtual machine and installing a Linux distribution on it. This will also fammiliarize you with the different aspects of a Linux installation.
\end{frame}

\begin{frame}[fragile]{Virtualbox}

  I personally use Linux Mint 17.2, but you can (and should!) try other distributions.

  Lots of different desktop versions: Cinnamon, MATE, GNOME 3, Unity, KDE...  
  
\end{frame}

\end{document}
