\documentclass[10pt]{beamer}

\usetheme[progressbar=frametitle]{metropolis}
\usepackage{appendixnumberbeamer}

\usepackage{siunitx}
\usepackage{booktabs}
\usepackage[scale=2]{ccicons}

\usepackage{hyperref}

\usepackage{pgfplots}
\usepgfplotslibrary{dateplot}

\usepackage{xspace}
\newcommand{\themename}{\textbf{\textsc{metropolis}}\xspace}

\title{Lecture 2}
\subtitle{ASTR 288P}
\date{2017/09/08}
\author{Sean Griffin}
\institute{UMCP / NASA GSFC}
%\titlegraphic{\vfill\includegraphics[height=1cm]{figures/University_of_Maryland_Seal.png}}

\newcommand{\ie}{\emph{i.e.}}
\newcommand{\eg}{e.g.}
\begin{document}

\maketitle

\begin{frame}[fragile]{Virtualbox}

  Virtualbox is a program that allows you to create and run virtual computeres within your host computer.

  This is a handy alternative to dual-booting if you want to use both Linux and Windows (e.g. if you're a hardcore gamer but need *nix to do analysis for work).

  You are effectively running an entire computer inside your OS, so performance can be an issue (particularly graphical stuff, although things are getting better).

  We'll go through the process of creating a virtual machine and installing a Linux distribution on it. This will also fammiliarize you with the different aspects of a Linux installation.
\end{frame}

\begin{frame}[fragile]{Virtualbox}

  I personally use Linux Mint 18.2, but you can (and should!) try other distributions.

  Lots of different desktop versions: Cinnamon, MATE, GNOME 3, Unity, KDE...  
  
  Something like LXDE Mint will probably be good on a low-powered machine.
  
\end{frame}


\begin{frame}[fragile]{What did we cover today?}
UNIX:
	\begin{itemize}
        \item Making files, changing permissions 
        \item Scripting in bash
	\end{itemize}
\end{frame}


\end{document}
